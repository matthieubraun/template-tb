\documentclass[
    iai, % Saisir le nom de l'institut rattaché
    eai, % Saisir le nom de l'orientation
    confidential, % Décommentez si le travail est confidentiel
]{heig-tb}

\usepackage[nooldvoltagedirection,european,americaninductors]{circuitikz}

\signature{mbernasconi.svg} % Remplacer par votre propre signature vectorielle.

\makenomenclature
\makenoidxglossaries
\makeindex

\addbibresource{bibliography.bib}

\input{nomenclature}
\input{acronyms}
testtest
\input{glossary}
% Auteur du document (étudiant-e) en projet de Bachelor
\author{Matthieu Braun}

% Activer l'option pour l'accord du féminin dans le texte
\genre{male}

% Titre de votre travail de Bachelor
\title{Etude et mise en oeuvre de la commande d'un déversoir pour le réglage précis de la hauteur d'eau dans un réacteur nucléaire expérimental}

% Le sous titre est optionnel
\subtitle{Travail de Bachelor}

% Nom du professeur responsable
\teacher {Prof. M. Girardin (HEIG-VD)}

% Mettre à jour avec la date de rendu du travail
\date{\today}

% Numéro de TB
\thesis{7212}



\surroundwithmdframed{minted}

%% Début du document
\begin{document}
\selectlanguage{french}
\maketitle
\frontmatter
\clearemptydoublepage

%% Requis par les dispositions générales des travaux de Bachelor
\preamble
\authentification

%% Résumé / Résumé publiable / Version abrégée
\begin{abstract}
    \input{abstract}
\end{abstract}

%% Sommaire et tables
\clearemptydoublepage
{
    \tableofcontents
    \let\cleardoublepage\clearpage
    \listoffigures
    \let\cleardoublepage\clearpage
    \listoftables
    \let\cleardoublepage\clearpage
    \listoflistings
}

\printnomenclature
\clearemptydoublepage
\pagenumbering{arabic}

%% Contenu
\mainmatter
\chapter{Introduction}
\input{introduction.tex}
\input{examples.tex}

\chapter{Conclusion}
\input{conclusion.tex}

\clearpage
\printbibliography

\appendix
\appendixpage
\addappheadtotoc

%%if
\chapter{Première annexe}

Les annexes n'ont pas un contenu \underline{normatif} mais \underline{descriptif}. Tout contenu annexé ne doit pas être nécessaire à la bonne compréhension du travail.

Les annexes contiennent généralement :

\begin{itemize}
    \item les dessins mécaniques (mises en plan);
    \item les schémas électriques détaillés;
    \item des photographies du projet;
    \item des scripts et des extraits de code source;
    \item des documents techniques \pex \emph{datasheet};
    \item des développements mathématiques.
\end{itemize}
\section{Sous section}
\lipsum[1]
%%fi

\let\cleardoublepage\clearpage
\backmatter

\label{glossaire}
\printnoidxglossary
\label{index}
\printindex

% Le colophon est le dernier élément d'un document qui contient des notes de l'auteur concernant la mise en page et l'édition du document : il est parfaitement optionnel.
\input{colophon.tex}

\end{document}
